\documentclass[10pt,fleqn,3p]{elsarticle}
\usepackage{amsmath,amsfonts,amssymb,mathpazo,indentfirst}
\newcommand*{\md}[1]{\mathrm{d}#1}
\newcommand*{\tr}[1]{\mathrm{tr}#1}
\newcommand*{\ddfrac}[2]{\dfrac{\md#1}{\md#2}}
\newcommand*{\pfrac}[2]{\dfrac{\partial#1}{\partial#2}}
\begin{document}\pagestyle{empty}
\section{Basics}
The two-scalar formulation depends on hydrostatic stress $p$ and von Mises equivalent stress $q$.
\begin{gather}
p=\dfrac{\tr{\sigma}}{3}=\dfrac{I_1}{3},\qquad{}q=\sqrt{3J_2}=\sqrt{\dfrac{3}{2}s:s}=\sqrt{\dfrac{3}{2}}|s|.
\end{gather}
\section{Yield Function}
The yield function is defined as
\begin{gather}
F=\dfrac{\left(p-p_t+a\right)^2}{b^2}+\dfrac{q^2}{M^2}-a^2,
\end{gather}
where $p_t\geqslant0$ is the tensile yield strength, $b=1$ if $p_e\geqslant0$ and $b=\beta$ if $p_e<0$ with $p_e=p-p_t+a$ denotes relative stress to the origin of ellipse, $\beta$ is a constant that controls the shape of negative-wards half of yielding ellipse, $M$ is the ratio between two radii of yielding ellipse.
\section{Flow Rule}
The associative plasticity is assumed so that
\begin{gather}
\Delta\varepsilon^p=\Delta\gamma\pfrac{F}{\sigma}=\Delta\gamma\left(\dfrac{2p_e}{3b^2}I+\dfrac{3}{M^2}s\right)
\end{gather}
via the following relationship
\begin{gather*}
\ddfrac{q^2}{\sigma}=\dfrac{3}{2}\ddfrac{\left(s:s\right)}{\sigma}=\dfrac{3}{2}\left(\ddfrac{s}{\sigma}:s+s:\ddfrac{s}{\sigma}\right)=3s.
\end{gather*}
Accordingly,
\begin{gather}
\Delta\varepsilon^p_v=\Delta\gamma\dfrac{2p_e}{b^2},\qquad\Delta\varepsilon^p_d=\Delta\gamma\dfrac{3}{M^2}s,
\end{gather}
where $\Delta\varepsilon^p_v=\tr{\Delta\varepsilon^p}$ is the volumetric strain scalar and $\Delta\varepsilon^p_d$ is the deviatoric strain tensor.
\section{Hardening Rule}
The hardening variable $\alpha$ is defined as the volumetric strain $\varepsilon^p_v$ so that
\begin{gather}
\alpha=\varepsilon^p_v.
\end{gather}
The corresponding incremental form is then
\begin{gather}
\alpha-\alpha_n=\Delta\alpha=\Delta\varepsilon^p_v=\Delta\gamma\dfrac{2p_e}{b^2},\\
\alpha-\alpha_n-\Delta\gamma\dfrac{2p_e}{b^2}=0.
\end{gather}
The hardening rule is then defined as a function of $\alpha$,
\begin{gather}
a=a\left(\alpha\right)\geqslant0.
\end{gather}
\section{Residual}
By using the elastic relationship,
\begin{gather}
s=2G\varepsilon^e_d=2G\left(\varepsilon^{tr}_d-\Delta\varepsilon^p_d\right)=s^{tr}-\Delta\gamma\dfrac{6G}{M^2}s,\\
p=K\varepsilon^e_v=K\left(\varepsilon^{tr}_v-\Delta\varepsilon^p_v\right)=p^{tr}-K\left(\alpha-\alpha_n\right).
\end{gather}
Hence,
\begin{gather}
s=\dfrac{M^2}{M^2+6G\Delta\gamma}s^{tr},\qquad
q=\dfrac{M^2}{M^2+6G\Delta\gamma}q^{tr}.
\end{gather}

The governing residual equations for independent variables $x=\begin{bmatrix}\Delta\gamma&\alpha\end{bmatrix}^\mathrm{T}$ can be expressed as
\begin{gather}
R=\left\{\begin{array}{l}
\dfrac{p_e^2}{b^2}+\dfrac{q^2}{M^2}-a^2=0,\\[4mm]
\alpha-\alpha_n-\Delta\gamma\dfrac{2}{b^2}p_e=0.
\end{array}\right.
\end{gather}
where $p_e=p^{tr}-K\alpha+K\alpha_n-p_t+a$ and $q=\dfrac{M^2}{M^2+6G\Delta\gamma}q^{tr}$.
\section{Local Iteration}
The Jacobian can be formed accordingly.
\begin{gather}
J=\ddfrac{R}{x}=\begin{bmatrix}
\dfrac{-12GM^2q^{tr,2}}{(M^2+6G\Delta\gamma)^3}&\dfrac{2}{b^2}p_e(a'-K)-2aa'\\[4mm]
-\dfrac{2}{b^2}p_e&1-\Delta\gamma\dfrac{2}{b^2}(a'-K)
\end{bmatrix}.
\end{gather}
\section{Tangent Stiffness}
At local iteration, $\varepsilon^{tr}$ is fixed and $R$ is iterated out. Noting that in the global iteration, $\varepsilon^{tr}$ is also a variable that changes. If local iteration is converged, then $R=0$, so
\begin{gather}
\ddfrac{R}{\varepsilon^{tr}}=\pfrac{R}{\varepsilon^{tr}}+\pfrac{R}{x}\ddfrac{x}{\varepsilon^{tr}}=0,
\end{gather}
consequently,
\begin{gather}
\ddfrac{x}{\varepsilon^{tr}}=
\begin{bmatrix}
\ddfrac{\Delta\gamma}{\varepsilon^{tr}}\\[4mm]\ddfrac{\alpha}{\varepsilon^{tr}}
\end{bmatrix}
=-\left(\pfrac{R}{x}\right)^{-1}\pfrac{R}{\varepsilon^{tr}}=-J^{-1}\pfrac{R}{\varepsilon^{tr}}.
\end{gather}

Taking derivatives about $\varepsilon^{tr}$ gives
\begin{gather}
\pfrac{R}{\varepsilon^{tr}}=\begin{bmatrix}
\dfrac{2p_eK}{b^2}I+\dfrac{6G}{M^2+6G\Delta\gamma}s\\[4mm]
\dfrac{-2\Delta\gamma{}K}{b^2}I
\end{bmatrix}
\end{gather}

The stress can be expressed as
\begin{gather}
\sigma=s+pI=\dfrac{M^2}{M^2+6G\Delta\gamma}s^{tr}+\left(p^{tr}-K\left(\alpha-\alpha_n\right)\right)I.
\end{gather}
Direct differentiation gives
\begin{gather}
\ddfrac{\sigma}{\varepsilon}=\dfrac{M^2}{M^2+6G\Delta\gamma}\ddfrac{s^{tr}}{\varepsilon}+s^{tr}\otimes\ddfrac{\dfrac{M^2}{M^2+6G\Delta\gamma}}{\varepsilon}+I\otimes\ddfrac{\left(p^{tr}-K\left(\alpha-\alpha_n\right)\right)}{\varepsilon}\\
\ddfrac{\sigma}{\varepsilon}=\dfrac{2GM^2}{M^2+6G\Delta\gamma}I_d+KI\otimes{}I-KI\otimes\ddfrac{\alpha}{\varepsilon}-\dfrac{6GM^2}{\left(M^2+6G\Delta\gamma\right)^2}s^{tr}\otimes\ddfrac{\Delta\gamma}{\varepsilon},\\
\ddfrac{\sigma}{\varepsilon}=D^e-\dfrac{12G^2\Delta\gamma}{M^2+6G\Delta\gamma}I_d-KI\otimes\ddfrac{\alpha}{\varepsilon}-\dfrac{6G}{M^2+6G\Delta\gamma}s\otimes\ddfrac{\Delta\gamma}{\varepsilon}.
\end{gather}
\end{document}
