\documentclass[10pt,fleqn,3p]{elsarticle}
\usepackage{amsmath,amsfonts,amssymb,mathpazo}
\newcommand*{\md}[1]{\mathrm{d}#1}
\newcommand*{\ramp}[1]{\left\langle#1\right\rangle}
\newcommand*{\mT}{\mathrm{T}}
\newcommand*{\tr}[1]{\mathrm{tr}~#1}
\newcommand*{\dev}[1]{\mathrm{dev}~#1}
\newcommand*{\ddfrac}[2]{\dfrac{\md#1}{\md#2}}
\newcommand*{\pfrac}[2]{\dfrac{\partial#1}{\partial#2}}
\title{Gurson Model}\date{}\author{tlc}
\begin{document}\pagestyle{empty}
This document outlines the algorithm used in the \textbf{uniaxial} Gurson porous model.
\section{Basics}
The two-scalar formulation depends on hydrostatic stress $p$ and von Mises equivalent stress $q$.
\begin{gather*}
\sigma=s+pI,
\end{gather*}
where
\begin{gather*}
s=\dev{\sigma},\qquad{}p=\dfrac{\tr{\sigma}}{3}=\dfrac{I_1}{3},\qquad{}q=\sqrt{3J_2}=\sqrt{\dfrac{3}{2}s:s}=\sqrt{\dfrac{3}{2}}|s|.
\end{gather*}
For the uniaxial formulation, it is assumed that $\sigma_2=\sigma_3=0$, so
\begin{gather}
s=\dfrac{1}{3}\sigma_1\begin{bmatrix}2&-1&-1&0&0&0
\end{bmatrix},\qquad{}p=\dfrac{1}{3}\sigma_1,\qquad
q=\sigma_1.
\end{gather}
The following expression would be useful in the derivation of the corresponding terms.
\begin{gather*}
\ddfrac{q^2}{\sigma}=\dfrac{3}{2}\ddfrac{\left(s:s\right)}{\sigma}=\dfrac{3}{2}\left(\ddfrac{s}{\sigma}:s+s:\ddfrac{s}{\sigma}\right)=3s.
\end{gather*}
\section{Yield Function}
The yield function is chosen to be
\begin{gather*}
F=q^2+2q_1f\sigma_y^2\cosh\left(\dfrac{3}{2}\dfrac{q_2p}{\sigma_y}\right)-\sigma_y^2\left(q_3f^2+1\right)=0,
\end{gather*}
where $\sigma_y$ is a function of the equivalent plastic strain $\varepsilon^p_m$, $f$ is the void volume fraction, the remaining $q_1$, $q_2$ and $q_3=q_1^2$ are material constants. In the uniaxial case, it becomes
\begin{gather}
F=\sigma_1^2+2q_1f\sigma_y^2\cosh\left(\dfrac{q_2}{2}\dfrac{\sigma_1}{\sigma_y}\right)-\sigma_y^2\left(q_3f^2+1\right)=0.
\end{gather}
\section{Flow Rule}
The associated flow rule is used.
\begin{gather*}
\Delta\varepsilon^p=\Delta\gamma\pfrac{F}{\sigma}=\Delta\gamma\left(\pfrac{F}{q^2}\ddfrac{q^2}{\sigma}+\pfrac{F}{p}\ddfrac{p}{\sigma}\right)=\Delta\gamma\left(\pfrac{F}{q^2}3s+\pfrac{F}{p}\dfrac{1}{3}I\right)=\Delta\gamma\left(3s+q_1q_2f\sigma_y\sinh{}I\right).
\end{gather*}
For $\Delta\varepsilon^p_1$,
\begin{gather*}
\Delta\varepsilon^p_1=\Delta\gamma\pfrac{F}{\sigma_1}=\Delta\gamma\left(2\sigma_1+q_1q_2f\sigma_y\sinh\right).
\end{gather*}
The volumetric strain increment is
\begin{gather}
\Delta\varepsilon^p_v=\Delta\gamma3q_1q_2f\sigma_y\sinh=\dfrac{p^{tr}-p}{K}=\dfrac{\sigma_1^{tr}-\sigma_1}{3K}.
\end{gather}
\section{Hardening Rule}
The equivalent plastic strain $\varepsilon^p_m$ evolves based on the following expression.
\begin{gather*}
\left(1-f\right)\sigma_y\Delta\varepsilon^p_m=\sigma:\Delta\varepsilon^p=\Delta\gamma\left(2\sigma_1^2+q_1q_2f\sigma_1\sigma_y\sinh\right).
\end{gather*}
The change of volume fraction $f$ is governed by the following equation.
\begin{gather*}
\Delta{}f=\Delta{}f_g+\Delta{}f_n,
\end{gather*}
where
\begin{gather*}
\Delta{}f_g=\left(1-f\right)\Delta\varepsilon^p_v=\left(1-f\right)\dfrac{\sigma_1^{tr}-\sigma_1}{3K},\qquad
\Delta{}f_n=A_N\Delta\varepsilon^p_m=A_N\left(\varepsilon^p_m-\varepsilon^p_{m,k}\right),
\end{gather*}
with $A_N=\dfrac{f_N}{s_N\sqrt{2\pi}}\exp\left(-\dfrac{1}{2}\left(\dfrac{\varepsilon^p_m-\varepsilon_N}{s_N}\right)^2\right)$. The corresponding derivative is
\begin{gather*}
\ddfrac{A_N}{\varepsilon^p_m}=A_N\dfrac{\varepsilon_N-\varepsilon^p_m}{s_N^2}.
\end{gather*}
\section{Residual}
The variables $\Delta\gamma$, $\varepsilon^p_m$, $f$ and $\sigma_1$ are chosen to be the independent variables. The corresponding residual equations are
\begin{gather}
R=\begin{Bmatrix}
F\\E\\V\\P
\end{Bmatrix}=\left\{\begin{array}{l}
\sigma_1^2+2q_1f\sigma_y^2\cosh-\sigma_y^2\left(q_3f^2+1\right)=0,\\
\left(1-f\right)\sigma_y\left(\varepsilon^p_m-\varepsilon^p_{m,k}\right)-2\Delta\gamma\sigma_1^2+\dfrac{\sigma_1^2-\sigma_1\sigma_1^{tr}}{9K}=0,\\
f-f_k+\left(1-f\right)\left(\dfrac{\sigma_1-\sigma_1^{tr}}{3K}\right)-A_N\left(\varepsilon^p_m-\varepsilon^p_{m,k}\right)=0,\\[4mm]
\sigma_1-\sigma_1^{tr}+9K\Delta\gamma{}q_1q_2f\sigma_y\sinh=0.
\end{array}\right.
\end{gather}
The Jacobian can be computed via the partial derivatives,
\begin{gather*}
\pfrac{F}{\Delta\gamma}=0,\qquad
\pfrac{F}{\varepsilon^p_m}=\left(4q_1f\sigma_y\cosh-q_1q_2f\sigma_1\sinh-2\sigma_y\left(q_3f^2+1\right)\right)\sigma_y',\\
\pfrac{F}{f}=2\sigma_y^2\left(q_1\cosh-q_3f\right),\qquad
\pfrac{F}{\sigma_1}=2\sigma_1+q_1q_2f\sigma_y\sinh,\\
\pfrac{E}{\Delta\gamma}=-2\sigma_1^2,\qquad
\pfrac{E}{\varepsilon^p_m}=\left(1-f\right)\left(\sigma_y'\left(\varepsilon^p_m-\varepsilon^p_{m,k}\right)+\sigma_y\right),\\
\pfrac{E}{f}=-\sigma_y\left(\varepsilon^p_m-\varepsilon^p_{m,k}\right),\qquad
\pfrac{E}{\sigma_1}=-4\Delta\gamma\sigma_1+\dfrac{2\sigma_1-\sigma_1^{tr}}{9K},\\
\pfrac{V}{\Delta\gamma}=0,\qquad
\pfrac{V}{\varepsilon^p_m}=A_N\dfrac{\varepsilon^p_m-\varepsilon_N}{s_N^2}\left(\varepsilon^p_m-\varepsilon^p_{m,k}\right)-A_N,\qquad
\pfrac{V}{f}=1-\dfrac{\sigma_1-\sigma_1^{tr}}{3K},\qquad
\pfrac{V}{\sigma_1}=\dfrac{1-f}{3K},\\
\pfrac{P}{\Delta\gamma}=9Kq_1q_2f\sigma_y\sinh,\qquad
\pfrac{P}{\varepsilon^p_m}=9K\Delta\gamma{}q_1q_2f\left(\sinh-\dfrac{q_2\sigma_1}{2\sigma_y}\cosh\right)\sigma_y',\\
\pfrac{P}{f}=9K\Delta\gamma{}q_1q_2\sigma_y\sinh,\qquad
\pfrac{P}{\sigma_1}=1+\dfrac{9}{2}K\Delta\gamma{}q_1q_2^2f\cosh.
\end{gather*}
The corresponding derivatives about the trial strain $\varepsilon_1^{tr}$ are
\begin{gather*}
\pfrac{F}{\varepsilon_1^{tr}}=0,\qquad
\pfrac{E}{\varepsilon_1^{tr}}=-\dfrac{E}{9K}\sigma_1,\qquad
\pfrac{V}{\varepsilon_1^{tr}}=\dfrac{\left(f-1\right)E}{3K},\qquad
\pfrac{P}{\varepsilon_1^{tr}}=-E.
\end{gather*}
\end{document}
